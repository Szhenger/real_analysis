\documentclass{amsart}
\usepackage{amsmath, amssymb, amsthm, enumitem}

\theoremstyle{definition}
\newtheorem{problem}{Problem}

\title{18.100B Problem Set 1}
\author{Shuo Zheng}
\date{December 18, 2024}

\begin{document}

\maketitle

\begin{problem}
    Let $m$ and $n$ be positive integers with no common factor. Prove that if $\sqrt{m/n}$ is rational, then $m$ and $n$ are perfect squares; that is, there exist integers $p$ and $q$ such that $m = p^2$ and $n = q^2$. (This is proved in Proposition 9 of Book X of \textit{Euclid's Elements})
\end{problem}

\begin{proof}
    Let $m \in \mathbb{N}$ and $n \in \mathbb{N}$ with no common factor. Suppose $\sqrt{m/n} \in \mathbb{Q}$. Then $\exists ! M,N \in \mathbb{N}$ such that $M$ and $N$ have no common factor and 
    \[     
    \sqrt{\frac{m}{n}} = \frac{M}{N} \implies \frac{m}{n} = \left(\frac{M}{N}\right)^2 = \frac{M^2}{N^2} \implies mN^2 = nM^2;     
    \] 
    hence, $M^2 \vert mN^2$ and $N^2 \vert nM^2$. By the Fundamental Theorem of Arithmetic, there are unique primes $p_1,...,p_k \in \mathbb{N}$ and exponents $r_1,...,r_k \in \mathbb{N}$ such that $M = p_1^{r_1} \cdot \cdot \cdot p_k^{r_k}$. Inductively, assume that $p_1^{2r_1} \cdot \cdot p_{k}^{2r_k} \vert m$. Thus,
    \[
    p_{k+1}^{2r_{k+1}} = \frac{M^2}{p_1^{2r_1} \cdot \cdot \cdot p_{k}^{2r_k}} \bigg \vert \frac{mN^2}{p_1^{2r_1} \cdot \cdot \cdot p_{k}^{2r_k}};
    \]
    however, $M$ and $N$ have no common factor, so
    \[
    p_{k+1}^{2r_{k+1}} \bigg \vert \frac{m}{p_1^{2r_1} \cdot \cdot \cdot p_{k}^{2r_k}} \implies M^2 = p_1^{2r_1} \cdot \cdot \cdot p_{k+1}^{2r_{k+1}} \vert m.    
    \]
    Similarly, we can obtain $N^2 \vert n$. Thus, $M^2 \vert m$ and $N^2 \vert n$, i.e. $\exists p,q \in \mathbb{N}$ such that $m = pM^2$ and $n = qN^2$; hence,
    \[
    mN^2 = nM^2 \implies pM^2N^2 = qM^2N^2 \implies p = q.
    \]
    Since $m$ and $n$ have no common factor, 
    \[
    p = q = 1 \implies m = M^2 \land n = N^2.
    \]
    Indeed, $m$ and $n$ are perfect squares.
\end{proof}

\begin{problem}
    Let $A$ and $B$ be two disjoint sets. Suppose further that $\vert A \vert = \vert \mathbb{R} \vert$ and that $\vert B \vert = \vert \mathbb{N} \vert$ (i.e. the set $B$ is countable). Show that $\vert A \cup B \vert = \vert \mathbb{R} \vert$.
\end{problem}

\begin{proof}
    Let $A$ and $B$ be disjoint sets. Suppose $\vert A \vert = \vert \mathbb{R} \vert$ and $\vert B \vert = \vert \mathbb{N} \vert$. Then $\exists$ bijective functions $f: A \to \mathbb{R}$ and $g: B \to \mathbb{N}$. Make a function $h: A \cup B \to \mathbb{R}$ such that
    \[
    h(x) = \begin{cases} 
      2f(x) - 1 & x \in A \land f(x) \in \mathbb{N} \\
      2g(x) & x \in B \\
      f(x) & x \in A \land f(x) \in \mathbb{R} \setminus \mathbb{N}
    \end{cases}.
    \]
    We want to show that $h$ is bijective ($1-1$ and onto). Assume $h(x) = h(y)$. Since $f$ and $g$ are $1-1$, 
    \[
    h(x) \in \{2n - 1 : n \in \mathbb{N}\} \implies 2f(x) - 1 = 2f(y) - 1 \implies f(x) = f(y) \implies x = y
    \]
    and 
    \[
    h(x) \in \{2n : n \in \mathbb{N}\} \implies 2g(x) = 2g(y) \implies g(x) = g(y) \implies x = y
    \]
    and 
    \[
    h(x) \in \mathbb{R} \setminus \mathbb{N} \implies f(x) = f(y) \implies x = y.
    \]
    Thus, $x = y$. Assume $z \in \mathbb{R}$. Since $f$ and $g$ are onto, 
    \[
    z \in \{2n - 1 : n \in \mathbb{N}\} \implies \exists x \in A, f(x) = \frac{z + 1}{2} \in \mathbb{N} \implies h(x) = 2f(x) - 1 = z
    \]
    and 
    \[
    z \in \{2n : n \in \mathbb{N}\} \implies \exists x \in B, g(x) = \frac{z}{2} \in \mathbb{N} \implies h(x) = 2g(x) = z
    \]
    and 
    \[
    z \in \mathbb{R} \setminus \mathbb{N} \implies \exists x \in A, f(x) = z \in \mathbb{R} \setminus \mathbb{N} \implies h(x) = f(x) = z.
    \]
    Thus, $\exists x \in A \cup B$ with $h(x) = z$. Indeed, $\vert A \cup B \vert = \vert \mathbb{R} \vert$.
\end{proof}

\begin{problem}
    Fix $b > 1$.
    \begin{enumerate}[label = (\alph*)]
        \item  If $m$, $n$, $p$, $q$ are integers, $n>0$, $q>0$, and $r = m/n = p/q$, prove that 
        \[
        (b^m)^{\frac{1}{n}} = (b^p)^{\frac{1}{q}}.
        \]
        Hence it makes sense to define $b^r := (b^m)^{\frac{1}{n}}$. (How could it have failed to make sense?)
        \begin{proof}
            Let $m,n,p,q \in \mathbb{Z}$ with $n > 0$ and $q > 0$. If $r = m/n = p/q$, then $mq = pn$; hence,
            \[
            [(b^m)^{\frac{1}{n}}]^{nq} = (b^m)^q = b^{mq} = b^{pn} = (b^p)^n = [(b^p)^{\frac{1}{q}}]^{nq}.
            \]
            Given Theorem 1.21, $\exists ! x \in \mathbb{R}$ such that 
            \[
            b^{mq} = x^{nq} = b^{pn} \implies (b^m)^{\frac{1}{n}} = x = (b^p)^{\frac{1}{q}}.
            \]
            Thus, $(b^m)^{\frac{1}{n}} = (b^p)^{\frac{1}{q}}$, i.e. defining
            \[
            b^r := (b^m)^{\frac{1}{n}}
            \]
            makes sense; else, $b^r$ cannot be well-defined.
        \end{proof}        
        \item Prove that $b^{r+s} = b^rb^s$ if $r$, $s$ are rational. 
        \begin{proof}
            Let $r \in \mathbb{Q}$ and $s \in \mathbb{Q}$. Then $\exists m,n,p,q \in \mathbb{Z}$ where $n > 0$ and $q > 0$ such that
            \[
            r = \frac{m}{n} \land s = \frac{p}{q} \implies r + s = \frac{m}{n} + \frac{p}{q} = \frac{mq+pn}{nq}.
            \]
            Given Theorem and Corollary 1.21, 
            \[
            (b^rb^s)^{nq} = b^{mq}b^{pn} = b^{mq+pn};
            \]
            hence,
            \[
            b^rb^s = (b^{mq+pn})^{\frac{1}{nq}} = b^{r+s}.
            \]
        \end{proof}        
        \item If $x$ is real, define $B(x)$ to be the set of all numbers $b^t$, where $t$ is rational and $t \leq x$. Prove that
        \[
        b^r = \sup B(r)        
        \]
        when $r$ is rational. Hence it makes sense to \textit{define}
        \[
        b^x := \sup B(x)
        \]
        for every real $x$.
        \begin{proof}
            Let $x \in \mathbb{R}$. Define the set
            \[
            B(x) := \{b^t \in \mathbb{R} : t \in \mathbb{Q} \ \text{and} \ t \leq x\}.
            \]
            Choose $r,q \in \mathbb{Q}$. If $r \geq q$, then $\exists m,n \in \mathbb{Z}$ such that $n >0$ and
            \[
            \frac{m}{n} = r - q \geq 0.
            \]
            Given that $b > 1$, $b^m \geq 1$; however, 
            \[
            0 < (b^m)^{\frac{1}{n}} = b^{r-q} < 1 \implies 0 < b^m < 1,
            \]
            a contradiction, i.e. $b^{r-q} \geq 1$. Hence, $b^r = b^{r-q}b^q \geq b^q$, i.e. $b^r$ is an upper bound of $B(r)$. Note that $b^r \in B(r)$. Thus, $b^r = \sup B(r)$, i.e. defining
            \[
            b^x := \sup B(x)
            \]
            makes sense.
        \end{proof}
        \item Prove that $b^{x+y} = b^xb^y$ for all real $x$ and $y$.
        \begin{proof}
            Let $x \in \mathbb{R}$ and $y \in \mathbb{R}$. Choose $r,s \in \mathbb{Q}$. If $r \leq x$ and $s \leq y$, then $r+s \leq x+y$; hence,
            \[
            b^rb^s = b^{r+s} \leq b^{x+y}.
            \]
            Thus, 
            \[
            b^r \leq \frac{b^{x+y}}{b^s} \implies b^x \leq \frac{b^{x+y}}{b^s};
            \]
            else, $b^x \neq \sup B(x)$. Similarly, we have that
            \[
            b^s \leq \frac{b^{x+y}}{b^x} \implies b^y \leq \frac{b^{x+y}}{b^x};
            \]
            otherwise, $b^y \neq \sup B(y)$. Thus, $b^xb^y \leq b^{x+y}$; hence, either $b^xb^y < b^{x+y}$ or $b^xb^y = b^{x+y}$. Suppose, for obtaining a contradiction, that $b^xb^y < b^{x+y}$. Thus,  $\exists t \in \mathbb{Q}$ such that $t < x+y$ and
            \[
            b^xb^y < b^t < b^{x+y};
            \]
            otherwise, $b^{x+y} \neq \sup B(x+y)$. By Theorem 1.20($a$), $\exists N \in \mathbb{N}$ such that
            \[
            N(x+y-t) > 1;
            \]
            hence,
            \[
            t < x+y-\frac{1}{N}.
            \]
            Given Theorem 1.20($b$), $\exists r,s \in \mathbb{Q}$ such that
            \[
            x-\frac{1}{2N} \leq r \leq x \land y-\frac{1}{2N} \leq s \leq y;
            \]
            hence,
            \[
            x+y-\frac{1}{N} \leq r+s \leq x+y.
            \]
            Thus, $b^t < b^{r+s} = b^rb^s \leq b^xb^y$, which is a contradiction. Indeed, $b^xb^y = b^{x+y}$.
        \end{proof}    
    \end{enumerate}
\end{problem}

\begin{problem}
    Prove that no order can be defined in the complex field that turns it into an ordered field. (\textit{Hint:} $-1$ is a square.)
\end{problem}

\begin{proof}
    Suppose, for obtaining a contradiction, an order $<$ can be defined in $\mathbb{C}$ making it into an ordered field. By Proposition 1.18, 
    \[
    -1 = i^2 \geq 0 \implies 0 \leq 1 = 1 + 0 \leq 1 + (-1) = 0
    \]
    hence,
    \[
    0 = 1,
    \]
    a contradiction.
\end{proof}

\begin{problem}
    Prove that 
    \[
    \vert x+y \vert^2 + \vert x-y \vert^2 = 2\vert x \vert^2 + 2\vert y \vert^2
    \]
    if $x,y \in \mathbb{R}^n$. Interpret this geometrically, as a statement about parallelograms.
\end{problem}

\begin{proof}
    Let $x \in \mathbb{R}^n$ and $y \in \mathbb{R}^n$. Then $\exists x_1,....,x_n,y_1,...,y_n \in \mathbb{R}$,
    \[
    x = (x_1,...,x_n) \land y = (y_1,...,y_n); 
    \]
    hence,
    \begin{align*}
        \vert x+y \vert^2 + \vert x-y \vert^2 &= \sum_{i = 1}^{n}(x_i+y_i)^2 + \sum_{i = 1}^{n}(x_i-y_i)^2 \\
        &= \sum_{i = 1}^{n}(x_i^2 + 2x_iy_i + y_i^2) + \sum_{i = 1}^{n}(x_i^2 - 2x_iy_i + y_i^2) \\
        &= \sum_{i = 1}^{n}(2x_i^2 + 2y_i^2) \\
        &= 2\sum_{i = 1}^{n}x_i^2 + 2\sum_{i = 1}^{n}y_i^2 = 2\vert x \vert^2 + 2\vert y \vert^2.
    \end{align*}
    Geometrically, we interpret vectors $x$ and $y$ as forming a parallelogram; hence, the sum of the diagonal length squared $\vert x+y \vert^2$ and anti-diagonal length squared $\vert x-y \vert^2$ is the sum of the side lengths squared $2\vert x \vert^2$ and $2\vert y \vert^2$. 
\end{proof}

\begin{problem}
    Suppose $f: \mathbb{R} \to \mathbb{R}$ is a function such that for all real numbers $x$ and $y$ the following two equations hold
    \begin{align}
        f(x + y) &= f(x) + f(y), \\
        f(xy) &= f(x)f(y).
    \end{align}
    \textbf{Claim:} $f(x) = 0$ for all $x$ or $f(x) = x$ for all $x$. \\
    Prove this claim using the following steps:
    \begin{enumerate}[label = (\alph*)]
        \item Prove that $f(0) = 0$ and that $f(1) = 0$ or $1$.
        \item  Prove that $f(n) = nf(1)$ for every integer $n$ and then that $f(n/m) = (n/m)f(1)$ for all integers $n$, $m$ such that $m \neq 0$. Conclude that either $f(q) = 0$ for all rational numbers $q$ or $f(q) = q$ for all rational numbers $q$.
        \item Prove that $f$ is increasing, that is to say that $f(x) \geq f(y)$ whenever $x \geq y$ for any real numbers $x$ and $y$.
        \item Prove that if $f(1) = 0$ then $f(x) = 0$ for all real numbers $x$. Prove that if $f(1) = 1$ then $f(x) = x$ for all real numbers $x$.        
    \end{enumerate}
\end{problem}

\begin{proof}
    Let $x \in \mathbb{R}$ and $y \in \mathbb{R}$.
    \begin{enumerate}[label = (\alph*)]
        \item If $x = y = 0$, then 
        \[
        f(x) + f(y) = f(x+y) \implies f(0) + f(0) = f(0) \implies f(0) = 0.
        \]
        If $x = y = 1$, then
        \begin{align*}
        f(xy) = f(x)f(y) \implies f(1) = f(1)^2 &\implies f(1)(1-f(1)) = 0 \\
        &\implies f(1) = 0 \lor 1.
        \end{align*}
        If $x = y$, then
        \[
        f(xy) = f(x)f(y) \implies f(x^2) = f(x)^2.
        \]
        If $x+y = 0$, then $y = -x$ and 
        \begin{align*}
        f(x) + f(y) = f(x+y) &\implies f(x) + f(-x) = f(0) = 0 \\
        &\implies f(-x) = -f(x).            
        \end{align*}    
        \item Suppose, for induction, that 
        \[
        \forall k \in \mathbb{Z}, f(k) = kf(1).
        \]
        Then 
        \[
        f(k+1) = f(k) + f(1) = kf(1) + f(1) = (k+1)f(1), 
        \]
        and 
        \[
        f(k-1) = f(k) + f(-1) = kf(1) - f(1) = (k-1)f(1);
        \]
        hence,
        \[
        \forall n \in \mathbb{Z}, f(n) = nf(1). 
        \]
        Thus,
        \begin{align*}
            m \in \mathbb{Z} \setminus \{0\} &\implies f(n) = f\left(\frac{n}{m}\right)f(m) \\
            &\implies nf(1) = f\left(\frac{n}{m}\right)mf(1) \\
            &\implies \frac{n}{m}f(1) = f\left(\frac{n}{m}\right)f(1) = f\left(\frac{n}{m}\right) \\ 
            &\implies f\left(\frac{n}{m}\right) = \frac{n}{m}f(1);
        \end{align*}
        hence, $f(q) = 0$ for all $q \in \mathbb{Q}$ or $f(q) = q$ for all $q \in \mathbb{Q}$.
        \item If $x \geq y$, then $x-y \geq 0$; hence, $\exists t \in \mathbb{R}$ such that $x-y = t^2$ (Theorem 1.21). Thus,
        \[
        f(x) - f(y) = f(x-y) = f(t^2) = f(t)^2 \geq 0 \implies f(x) \geq f(y);
        \]
        hence, $f$ is increasing.
        \item Given $n \in \mathbb{N}$, $\exists p,q \in \mathbb{Q}$ such that 
        \[
        x - \frac{1}{n} \leq p \leq x \leq q \leq x + \frac{1}{n}
        \]
        (Theorem 1.20($b$)). If $f(1) = 0$, then 
        \[
        0 = f(p) \leq f(x) \leq f(q) = 0 \implies f(x) = 0
        \]
        (because $f$ is increasing).
        If $f(1) = 1$, then 
        \[
        x - \frac{1}{n} \leq p = f(p) \leq f(x) \leq f(q) = q \leq x + \frac{1}{n} \implies f(x) = x
        \]
        (because $n$ is arbitrary).
    \end{enumerate}
    Thus, $f(x) = 0$ or $f(x) = x$ and $f(y) = 0$ or $f(y) = y$, as desired.
\end{proof}
\end{document}
